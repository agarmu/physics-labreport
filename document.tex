\documentclass[12pt, letterpaper]{article}
\usepackage{layout,geometry}
\geometry{
    letterpaper,
    inner=1in,
    top=1in,
    outer=1in,
    bottom=1in,
    marginparwidth=0.5in
}
\usepackage{graphicx}
\usepackage{import,xifthen,pdfpages,transparent}
\usepackage{float}
\newcommand{\incfig}[2][1]{%
    \def\svgwidth{#1\columnwidth}
    \import{./figures/}{#2.pdf_tex}
}

\usepackage{fontspec,xltxtra,xunicode}
\defaultfontfeatures{Mapping=tex-text}
\setromanfont[Mapping=tex-text]{Libre Baskerville}
\setsansfont[Scale=MatchLowercase,Mapping=tex-text]{Montserrat}
\setmonofont[Scale=MatchLowercase]{JetBrainsMono Nerd Font Mono}

\usepackage{booktabs, longtable}
\usepackage[shortlabels]{enumitem}
\usepackage{emptypage}
\usepackage{subcaption}
\usepackage{multirow, multicol}

\usepackage{amsmath, amsfonts, mathtools, amsthm, amssymb}
\usepackage{cancel}
\usepackage{bm}

\usepackage{systeme} % Systems of equations

\let\implies\Rightarrow
\let\impliedby\Leftarrow
\let\iff\Leftrightarrow

% horizontal rule
\newcommand\hr{
    \noindent\rule[0.5ex]{\linewidth}{0.5pt}
}


% hide parts
\newcommand\hide[1]{}



% Système International Units
\usepackage{siunitx}
\sisetup{locale = US}

% colored box environments

\usepackage{tikz}
\usepackage{tikz-cd}
\usetikzlibrary{intersections, angles, quotes, calc, positioning}
\usetikzlibrary{arrows.meta}
\usepackage{pgfplots}
\pgfplotsset{compat=1.13}
\tikzset{
    force/.style={thick, {Circle[length=2pt]}-stealth, shorten <=-1pt}
}


\title{Title}
\author{Author}
\date{}  %Put date here

\begin{document}
\maketitle

The abstract is a concise summary of the lab report. A good abstract should state the purpose, procedure, principal 
results, conclusion, and implications of the lab in a single paragraph that is generally 100 to 200 words in length 
(use your word processor’s word count tool to check length).
\section{Introduction}

A complete scientific lab report has an introduction that gives the context for the experiment, the background theory, 
and a description of the experimental procedure and equipment used. 

\section{Data and Results}

The results of your experiment must be well organized and easy to read. When appropriate, tables or graphs should 
be used to present data and results. Graphs must be properly constructed (with a computer or by hand, as directed) 
with descriptive titles, labeled axes with relevant units, and calculated parameters properly interpreted (e.g. What do 
the slope and intercept represent?) All measured values must have four critical parts:
\begin{enumerate}
    \item A label (word or symbol) that clearly identifies the measured value
    \item The numerical value for the measurement (rounded to be consistent with the uncertainty)
    \item A reasonable estimate of the uncertainty associated with the measurement
    \item An appropriate unit of measure (SI units are usually preferred)
\end{enumerate}


Sample calculations, including an analysis of the experimental uncertainties, should be shown for any derived or 
calculated values as appropriate. Your original, unaltered data sheet must be included either in the data section or 
as an appendix.
\section{Discussion}

In the discussion section, summarize the results you obtained, and then discuss any discrepancies between your 
results and what was expected according to the given theoretical predictions or your own hypotheses. Did the 
experimental results agree with your predictions or the findings from other lab groups? If not, what is the most likely 
reason for the discrepancy? Remember to consider the uncertainty of your results when determining agreement. 
Identify the primary source of error in your results and justify your answer based on your uncertainty estimates. 
(Note: General statements without justification and explanation are not acceptable. “Human error” is not an 
acceptable source of error!) How could you improve the quality of your measurements with the available equipment? 
What did you learn or discover from this lab? The discussion section for most labs should be about one to two 
pages in length. Remember that your discussion will be graded on the quality of your explanations, not the quantity. 

\end{document}